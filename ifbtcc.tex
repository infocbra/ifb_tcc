%% IFBTcc Latex Template Version
%% 	based on a fork of : RiSE Latex Template
%%
%% IFBthesis latex template for thesis and dissertations
%% https://github.com/IFBmodels/tcc
%%
%% (c) 2017 Rafael de Campos Passos (rcpassos@ieee.org)
%%
%% This document was initially based on RiSE Latex template, from Yguaratã
%% Cerqueira Cavalcanti
%%
%% GENERAL INSTRUCTIONS
%%
%% We strongly recommend you to compile your documents using pdflatex command.
%% It is also recommend use the texlipse plugin for Eclipse to edit your documents.
%%
%% Options for \documentclass command:
%%         * Idiom
%%           pt   - Portguese (default)
%%           en   - English
%%
%%         * Text type
%%           bsc  - B.Sc. Thesis
%%           msc  - M.Sc. Thesis (default)
%%           qual - PHD qualification (not tested yet)
%%           prop - PHD proposal (not tested yet)
%%           phd  - PHD thesis
%%
%%         * Media
%%           scr  - to eletronic version (PDF) / see the users guide
%%
%%         * Pagination
%%           oneside - unique face press
%%           twoside - two faces press
%%
%%		   * Line spacing
%%           singlespacing  - the same as using \linespread{1}
%%           onehalfspacing - the same as using \linespread{1.3}
%%           doublespacing  - the same as using \linespread{1.6}
%%
%% Reference commands. Use the following commands to make references in your
%% text:
%%          \figref  -- for Figure reference
%%          \tabref  -- for Table reference
%%          \eqnref  -- for equation reference
%%          \chapref -- for chapter reference
%%          \secref  -- for section reference
%%          \appref  -- for appendix reference
%%          \axiref  -- for axiom reference
%%          \conjref -- for conjecture reference
%%          \defref  -- for definition reference
%%          \lemref  -- for lemma reference
%%          \theoref -- for theorem reference
%%          \corref  -- for corollary reference
%%          \propref -- for proprosition reference
%%          \pgref   -- for page reference
%%
%%          Example: See \chapref{chap:introduction}. It will produce
%%                   'See Chapter 1', in case of English language.
%%
%% Citation commands:
%%          \citet (from natbib) -- To cite a reference as part of the narrative
%%          \citep (from natbib) -- To cite a reference between parenthesis
%%          citationblock environment -- To produce direct citation blocks according to the ABNT

\documentclass[pt,twoside,onehalfspacing,bsc]{ifbclass/ifbclass}

  \usepackage{colortbl}
  \usepackage{color}
  \usepackage[table]{xcolor}
  \usepackage{microtype}
  \usepackage{bibentry}
  \usepackage{subfigure}
  \usepackage{multirow}
  \usepackage{rotating}
  \usepackage{booktabs}
  \usepackage{pdfpages}
  \usepackage{caption}
  \usepackage{lipsum}
  \usepackage{sectsty}
  
  %% Set the language used in your code in the block above
  
  \captionsetup[table]{position=top,justification=centering,width=.85\textwidth,labelfont=bf,font=footnotesize}
  \captionsetup[lstlisting]{position=top,justification=centering,width=.85\textwidth,labelfont=bf,font=footnotesize}
  \captionsetup[figure]{position=bottom,justification=centering,width=.85\textwidth,labelfont=bf,font=footnotesize}
  
  %% Chapter and (Sub)Section fonts must be same size as text (12)
  \sectionfont{\fontsize{12}{15}\selectfont}
  \subsectionfont{\fontsize{12}{15}\selectfont}
  \subsubsectionfont{\fontsize{12}{15}\selectfont}
  
  %% Change the following pdf author attribute name to your name.
  \usepackage[linkcolor=black,
              citecolor=black,
              urlcolor=black,
              colorlinks,
              pdfpagelabels,
              pdftitle={Rise Thesis Template (ABNT)},
              pdfauthor={Rise Thesis Template (ABNT)},
              breaklinks=true]{hyperref}
  
  \address{BRASÍLIA}
  
  \universitypt{Instituto Federal de Brasília}
  \universityen{Federal Institute of Brasilia}
  
  \campus{Campus Taguatinga}
  
  \departmentpt{Departamento de Computação}
  \departmenten{Computer Department}
  
  \programpt{Bacharelado em Ciência da Computação}
  \programen{Bachelors in Computer Science}
  
  \majorfieldpt{Ciência da Computação}
  \majorfielden{Computer Science}
  
  \title{Título do Trabalho }
  
  \date{2017}
  
  \author{Nome completo do Autor}
  \adviser{Nome completo do Orientador}
  \coadviser{Nome dompleto do co-orientador }
  
  % Macros (defines your own macros here, if needed)
  \def\x{\checkmark}
  %\let\lstlistoflistings\origlstoflistings
  \begin{document}
  
  \frontmatter
  
  \frontpage
  
  \presentationpage
  
  \begin{fichacatalografica}
    \FakeFichaCatalografica % Comment this line when you have the correct file
  %     \includepdf{fig_ficha_catalografica.pdf} % Uncomment this
  \end{fichacatalografica}
  
  \banca
  
  \begin{dedicatory}
  I dedicate this thesis to all my family, friends and professors who gave me the
  necessary support to get here.
  \end{dedicatory}
  
  \acknowledgements
  Agradeço ao meu orientador Prof. Dr. Nome do Orientador, pela sabedoria com que me guiou nesta trajetória.

Aos meus colegas de sala.

A Secretaria do Curso, pela cooperação.

Gostaria de deixar registrado também, o meu reconhecimento à minha família, pois acredito que sem o apoio deles seria muito difícil vencer esse desafio. 

Enfim, a todos os que por algum motivo contribuíram para a realização desta pesquisa.

  
  \begin{epigraph}[]{Márcio de Deus}
  When one finds a hard problem, the more complicated it is, the more one ought to work towards enlightening it's solution.
  \end{epigraph}
  
  \resumo
  % Escreva seu resumo no arquivo resumo.tex
  {\parindent0pt
    SOBRENOME, Prenome do Autor do Trabalho. Título do trabalho: subtítulo (se houver).  2018. 65 f. 
Trabalho de Conclusão de Curso (Graduação) – Tecnólogo em Sistemas para Internet. 
Instituto Federal de Brasília – Campus Brasília. Brasília/DF, 2018.
\vspace{1cm}

Elemento obrigatório, constituído de uma sequência de frases concisas e objetivas,
fornecendo uma visão rápida e clara do conteúdo do estudo. O texto deverá conter no
máximo 500 palavras e ser antecedido pela referência do estudo, com exceção do resumo
inserido no próprio documento. Também, não deve conter citações. O resumo deve ser redigido
em parágrafo único, espaçamento simples e seguido das palavras representativas do conteúdo
do estudo, isto é, palavras-chave, em número de três a cinco, separadas entre si por ponto e
finalizadas também por ponto. Usar o verbo na terceira pessoa do singular, com linguagem
impessoal (pronome SE), bem como fazer uso, preferencialmente, da voz ativa.

\begin{keywords}
Primeira palavra. Segunda palavra. Terceira palavra. Quarta palavra. Quinta-palavra.
\end{keywords}

  }
  
  \abstract
  % Write your abstract in a file called abstract.tex
  {\parindent0pt
    \input{text/abstract}
  }
  
  % List of figures
  \listoffigures
  
  % List of Codes
  \lstlistoflistings
  
  % List of tables
  \listoftables
  
  % List of acronyms
  % Acronyms manual: http://linorg.usp.br/CTAN/macros/latex/contrib/acronym/acronym.pdf
  \listofacronyms
  \input{text/acronyms}
  
  % Summary (tables of contents)
  \tableofcontents
  
  \mainmatter
  
  \include{text/introduction}
  \include{text/background}
  \chapter{Development}

\section{Introduction}

\lipsum[1]

\begin{code}[language=Python,caption=Python Fribonacci Code,label=code:frib]
from math import *

# define function
def analytic_fibonacci(n):
  sqrt_5 = sqrt(5);
  p = (1 + sqrt_5) / 2;
  q = 1/p;
  return int( (p**n + q**n) / sqrt_5 + 0.5 )

for i in range(1,31):
  print analytic_fibonacci(i)
\end{code}


This is a reference to Code \ref{code:frib} \ldots{}

\lipsum[1]

\begin{code}[language=C,caption=Hello World C Code,label=code:helloc]
#include<stdio.h>

main()
    {
        printf("Hello World");
    }
\end{code}


This is a reference to Code \ref{code:helloc} \ldots{}

\lipsum[1]

\begin{code}[language=Java,caption=Hello Java Code,label=code:helloj]
public class HelloWorld {

    public static void main(String[] args) {
        System.out.println("Hello, World");
    }
}
\end{code}


This is a reference to Code \ref{code:helloj} \ldots{}


\section{Section}

\lipsum[2-4]

\subsection{Subsection}

\lipsum[2-4]

  \include{text/conclusion}
  
  % References
  
  \begin{references}
    \bibliography{bib/references}
  \end{references}
  
  % Appendix
  
  \theappendix
  \include{appendix/mapping-study}
  
  \end{document}
  